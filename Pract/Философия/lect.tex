\documentclass{article}
\usepackage{graphicx} % Required for inserting images
\usepackage{amsmath}
\usepackage[english, russian] {babel}
\usepackage[utf8]{inputenc}
\usepackage[T2A]{fontenc}
\usepackage{minted}
\usepackage{float}
\usepackage{amssymb}

\title{Философия. Практика}
\author{silvia.lesnaia }


\begin{document}

\maketitle

\textbf{17.02.26}

Филосфоия - любовь к мудрости. 

Мудрость - умение воспользоваться знанием и опытом. 

\section{Разделы философии}

\textbf{Онтология} - учение о бытии, раздел философии, изучающие фундометнатльные рицницы бытия, 
его набилее общие сущности, категории, структуры и закономерности. 

\textbf{Гносеология} - раздел о природе познания. Его возмоожности в границах и формах.

\textbf{Логика} - философская дицсиплина, о законах и формах правильного мышления.

\textbf{Этика} - философское учение о морали и нравственности. 
Мораль - систмеаа идеальгых предствальний о прицнипах и нормах должного и желанного поведения.
Нравтсвеность - совокупность норм и правил которыми человек руквоводствуется в реальном поведении.

\textbf{Эстетика} - философское учение о сущности и формах прекрасного, в художественном творчестве, 
в природе и жизни, об искусстве как о форме общественного сознания. 

\textbf{Социальная и история философии} еще есть, ну тут все понятно из названия. 

\textbf{Аксиология} - раздел философии изучающий приороды ценностей их место в реальности и струкутру 
целостности мира. 


Д/З философия Индии и Китая. В рамках индийской брахманизма, джанизм, буддизм. Китайская - Инь-Янь, Джаон-Цзы
Мен-Цзы, Сюнь-Цзы и Моц-Цзы, Конфуциантсво. 
\end{document}